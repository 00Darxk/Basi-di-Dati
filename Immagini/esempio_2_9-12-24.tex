% !TEX TS-program = lualatex
\documentclass{standalone}
\usepackage{tikz}
\usepackage{pgf}
\usepackage{pgfcore}
\usepackage{pgfplots}
\usepackage{amsmath}
\usepackage{amsfonts}
\usepackage{contour}
\usepackage[normalem]{ulem}
\usepackage{cancel}
\usetikzlibrary{graphs,graphdrawing,calc}
\pgfplotsset{compat=1.18}
\renewcommand{\ULdepth}{1.8pt}
\contourlength{0.6pt}
\newcommand{\myuline}[1]{%
  \uline{\phantom{#1}}%
  \llap{\contour{white}{#1}}%
}
\begin{document}
\usegdlibrary{trees}
    \tikz[tree layout, nodes={thick}, inner sep=1pt]
    \node[align=center] {$R$(\myuline{Codice Corso}, \myuline{Anno}, Titolo, Crediti, Docente, Nome Docente, Semestre, Dipartimento, Indirizzo)\\Codice Corso $\to$ Titolo, Crediti, Dipartimento\\Codice Corso, Anno $\to$ Docente, Semestre\\Docente $\to$ Nome Docente\\Dipartimento $\to$ Indirizzo}
        child { node[align=center]{$R_1$(\myuline{Codice Corso}, Titolo, Crediti, Dipartimento)\\Codice Corso $\to$ Titolo, Crediti, Dipartimento} }
        child { node[align=center]{$R_2$(\myuline{Codice Corso}, \myuline{Anno}, Docente, Nome Docente, Semestre, Indirizzo)\\Codice Corso, Anno $\to$ Docente, Semestre\\Docente $\to$ Nome Docente\\\cancel{Dipartimento $\to$ Indirizzo}} 
            child { node[align=center] {$R_3$(\myuline{Docente}, Nome Docente)\\Docente $\to$ Nome Docente}}
            child { node[align=center] {$R_4$(\myuline{Codice Corso}, \myuline{Anno}, Docente, Semestre, Indirizzo)\\Codice Corso, Anno $\to$ Docente, Semestre}}
        } 
    ;
\end{document}
\documentclass{article}

\usepackage{cancel}
\usepackage{amsmath}
\usepackage[includehead,nomarginpar]{geometry}
\usepackage{graphicx}
\usepackage{amsfonts} 
\usepackage{verbatim}
\usepackage{mathrsfs}  
\usepackage{lmodern}
\usepackage{braket}
\usepackage{bookmark}
\usepackage{fancyhdr}
\usepackage{romanbarpagenumber}
\usepackage{minted}
%\usepackage{subfig}
\usepackage[italian]{babel}
%\usepackage{float}
%\usepackage{wrapfig}
%\usepackage[export]{adjustbox}
\allowdisplaybreaks

\setlength{\headheight}{12.0pt}
\addtolength{\topmargin}{-12.0pt}
\graphicspath{ {./Immagini/} }

\hypersetup{
    colorlinks=true,
    linkcolor=black,
}

\newsavebox{\tempbox} %{\raisebox{\dimexpr.5\ht\tempbox-.5\height\relax}}


\makeatother

\numberwithin{equation}{subsection}
\newcommand{\tageq}{\tag{\stepcounter{equation}\theequation}}
\AtBeginDocument{%
  \renewcommand{\figurename}{Fig.}
}
\fancypagestyle{link}{\fancyhf{}\renewcommand{\headrulewidth}{0pt}\fancyfoot[C]{Sorgente del file LaTeX disponibile al seguente link: \url{https://github.com/00Darxk/Basi-di-Dati/}}}

\begin{document}

\title{%
    \textbf{Basi di Dati}  \\ 
    \large Appunti delle Lezioni di Basi di Dati \\
    \textit{Anno Accademico: 2024/25}}
\author{\textit{Giacomo Sturm}}
\date{\textit{Dipartimento di Ingegneria Civile, Informatica e delle Tecnologie Aeronautiche \\
Università degli Studi ``Roma Tre"}}

\maketitle
\thispagestyle{link}

\clearpage


\pagestyle{fancy}
\fancyhead{}\fancyfoot{}
\fancyhead[C]{\textit{Basi di Dati - Università degli Studi ``Roma Tre"}}
\fancyfoot[C]{\thepage}
\pagenumbering{Roman}

\tableofcontents

\clearpage
\pagenumbering{arabic}

%% TODO aggiungere lezione 23, 27, 30 sett; 4, 7, 11 ott;

\section{Introduzione}

\clearpage

\section{Algebra Relazionale}

\clearpage

\section{Introduzione a SQL}

%% TODO divisione in sottosezioni

SQL è indifferente tra maiuscolo e minuscolo, ma è preferibile essere coerente con le scelte di sintassi effettuate. Inoltre è indifferente 
dall'indentazione, ma si preferisce inserire parentesi oppure si va a capo per aumentare la leggibilità dell'interrogazione. Il linguaggio lo 
interpreta cercando il nome della parola chiave, e gli argomenti dell'interrogazione.  

Le istruzioni che coinvolgono più relazioni, nel formato base, consiste in una parola chiave \verb|SELECT| seguita la lista dove operare, 
seguita dalla clausola \verb|FROM| ed in caso una condizione introdotta con \verb|WHERE|. 

\begin{minted}{SQL}
  SELECT ListaAttributi
  FROM ListaTabelle
  WHERE Condizione
\end{minted}

Essenzialmente realizza un prodotto cartesiano delle relazioni specificate nella clausola \verb|FROM|, in seguito viene effettuata un'operazione di selezione in base alla 
condizione specificata dalla parola chiave \verb|WHERE|, se si cercano solo certi attributi, si può effettuare una proiezione specificando la lista di attributi dopo 
la parola chiave \verb|SELECT|. In SQL bisogna specificare con la parola chiave \verb|DISTINCT| che si stanno cercando solo gli attributi diversi. 


In algebra relazionale è possibile scrivere interrogazioni equivalenti in modi diversi, in cui ci sono variazioni di efficienza, 
l'algebra è procedurale. In SQL invece il sistema si preoccupa dell'efficienza delle operazioni, è almeno in parte dichiarativo dove le interrogazioni 
possono essere scritte in modi diversi, ma alcune differenze presenti in algebra non emergono. 


Il sistema esegue selezione join ed un ulteriore proiezione, nella versione base di SQL. QUalche anno dopo venne introdotto il join esplicito, 
introducendo la possibilità di specificare i join nella clausola \verb|FROM| specificando l'argomento al posto di una lista di attributi, una lista 
di join effettuati su attributi, specificando la condizione di join dopo la clausola \verb|ON|. 

Date due relazioni contenente un attributo in comune, per realizzare una relazione di join su questo attributo in comune si specifica nella 
clausola \verb|SELECT| l'attributo in notazione puntata, altrimenti solleverebbe un errore poiché rappresenta un nome ambiguo. Anche nella 
condizione di join nella clausola \verb|FROM| bisogna specificare a chi appartiene l'attributo indicato, utilizzando la notazione puntata: 

\begin{minted}{SQL}
  SELECT Attributo1, Attributo2, Lista1.AttributoComune 
  --oppure anche Lista2.AttributoComune
  FROM Lista1 JOIN Lista2 ON Lista1.AttributoComune = Lista2.AttributoComune
\end{minted}

In SQL esiste il modo per effettuare join naturali specificando il nome dell'attributo non in notazione puntata, utilizzando la clausola \verb|USING|, 
ma è preferibile non usarlo per favorire la comprensione:

\begin{minted}{SQL}
  SELECT AttributoComune, Attributo1, Attributo2
  FROM Lista1 JOIN Lista2 USING AttributoComune
\end{minted}

Inoltre è possibile utilizzare più volte la stessa relazione in un'interrogazione, utilizzando un nome diverso, chiamati alias in SQL, specificando dopo il nome 
l'alias, oppure utilizzando la clausola \verb|AS|. In questo modo è possibile eliminare le ambiguità generate effettuando diversi join sulle stesse relazioni. 
\begin{minted}{SQL}
  Nome AS N
\end{minted}

Questo è utile per visualizzare campi dallo stesso nome, ridenominando gli attributi del risultato, oppure per facilitare la scrittura 
evitando nomi di relazioni molto lunghi. 

Esiste in SQL il join esterno, dove alcuni degli operandi partecipano solamente in parte, tramite la clausola \verb|LEFT|, \verb|RIGHT| o \verb|FULL| 
seguito da \verb|JOIN|. Inoltre è possibile inserire \verb|OUTER| per realizzare join equivalenti, ma queste funzioni non sono presenti nel servizio web SQLite, dove 
è possibile utilizzare solamente \verb|LEFT JOIN|. 

L'ordinamento del risultato è un altro fattore determinante, in base a cui si distinguono due ennuple 
o soluzioni tra di loro. Si può effettuare operazioni sulla target list e si può utilizzare la condizione \verb|LIKE| per identificare espressioni regolari, dove 
\verb|_| identifica un qualsiasi carattere e \verb|%| per qualsiasi sequenza di carattere, inseriti tra doppi apici \verb|" ... "|. 

Il contenuto delle basi di dati viene spesso aggregato, ma questo non è possibile in algebra relazionale. SQL prevede la possibilità di calcolare piccole elaborazioni 
a partire da insiemi di ennuple, di conteggio, minimo, massimo, media o totale. 

Queste operazioni vengono svolte da operatori aggregati quali \verb|COUNT| per contare tutti le righe in una relazione. Le funzioni aggregative 
lavorano anche con valori nulli. Bisogna specificare l'attributo di cui contare tutte le righe nella relazione tra parentesi tonde:
\begin{minted}{SQL}
  COUNT(Attributo)
\end{minted}
Altri operatori aggregati sono \verb|SUM|, \verb|AVG|, \verb|MAX| e \verb|MIN|. Si nota un'ulteriore utilità del valore nullo, poiché il sistema li 
riconosce come un valore non reale e non lo utilizza nel calcolo, al contrario di un sistema dove i valori nulli vengono codificati con 0 o -1. 

Esiste un'altra clausola \verb|GROUP BY|, insieme alle funzione aggregate, divide le ennuple di una relazione sulla base dell'attributo specificato, questo attributo deve 
essere presente anche nella target list. 
Ma in questo modo nel raggruppamento sotto l'attributo raggruppato, se è presente più di uno, sarà scelto uno a caso da visualizzare, su SQLite. 

Esiste un altra clausola \verb|HAVING| per definire una condizione su raggruppamenti, mentre la \verb|WHERE| si usa sulle singole ennuple. 

%% TODO problema di trovare il max (fine lezione 14 ott)


\end{document}